\documentclass{article}

\usepackage{mathtools}
\usepackage{bm}
\usepackage[margin=1.25in]{geometry}
\usepackage{courier}
\usepackage{color}
\usepackage{listings}
\usepackage{pdfpages}
\usepackage{enumerate}

\definecolor{dkgreen}{rgb}{0,0.6,0}
\definecolor{gray}{rgb}{0.5,0.5,0.5}

\newcommand{\blankpage}{
	\newpage
	\thispagestyle{empty}
	\mbox{}
	\newpage
}

\newcommand{\exspace}{\hspace{0.1in}}

\title{Optimization Methods, Final Exam}
\date{2015/12/16}
\author{Matthew Grasinger}

\begin{document}
	
\lstset{language=Matlab,
	keywords={break,case,catch,continue,else,elseif,end,for,function,
		global,if,otherwise,persistent,return,switch,try,while},
	basicstyle=\ttfamily,
	keywordstyle=\color{blue},
	commentstyle=\color{gray},
	stringstyle=\color{dkgreen},
	numbers=left,
	numberstyle=\tiny\color{red},
	stepnumber=1,
	numbersep=10pt,
	backgroundcolor=\color{white},
	tabsize=4,
	showspaces=false,
	showstringspaces=false}
	
\pagenumbering{gobble}
\maketitle
\blankpage
\tableofcontents
\blankpage
\pagenumbering{arabic}

\section{Problem 1} \label{sec:smallest_circle}

\subsection{Objective}

Find the center and radius of the smallest circle that encompasses the following points: $(0,1)$, $(1,0)$, $(1,1)$, $(0,0)$, and whose center lies along the line $y=2x$.

\subsection{Approach 1}

\begin{enumerate}[a)]

\item Express the problem as a constrained minimization problem.

The general equation of a circle can be given as,
\begin{equation} \label{eq:circle_general}
(x - c_x)^2 + (y - c_y)^2 = r^2,
\end{equation}
where $c_x$ and $c_y$ are the $x$ and $y$ coordinates of the center of the circle respectively, and $r$ is the radius of the circle.
The area of a circle is proportional to its radius squared.
Minimizing $r^2$ will be sufficient for finding the smallest circle that encompasses the given points.
By substituting the constraint on the center of the circle into Equation \ref{eq:circle_general}, we obtain the desired constrained minimization formulation,

\begin{align*}
\min_{r, c_x} \exspace & r^2, \\
\text{such that} \exspace & (x_i - c_x)^2 + (y_i - 2c_x)^2 - r^2 \le 0, \\
& \forall i = 1, 2, ..., n.
\end{align*}

\item Find the optimality conditions.

The KKT optimality conditions for the smallest circle problem are,

\begin{eqnarray*}
\boldsymbol{\nabla} f(\mathbf{x}) + \sum_{i}^{n} \mu_i \boldsymbol{\nabla} g_i(\mathbf{x}) &=& \mathbf{0},\\
\begin{bmatrix}2r \\ 0\end{bmatrix} + \sum_{i}^{n} \mu_i \begin{bmatrix}-2r \\ -2(x_i - c_x) - 4(y_i - 2c_x)\end{bmatrix} &=& \mathbf{0},
\end{eqnarray*}

and,

\begin{eqnarray*}
\sum_{i}^{n} \mu_i g_i(\mathbf{x}) &=& 0,\\
\sum_{i}^{n} \mu_i \left[(x_i - c_x)^2 + (y_i - 2c_x)^2 - r^2\right] &=& 0,
\end{eqnarray*}

and lastly,

\begin{equation*}
	\mathbf{\mu} = \mathbf{0}.
\end{equation*}

%The second order sufficient condition is that

%\begin{eqnarray*}
%	\mathbf{L}(\mathbf{x}) &=& \mathbf{F}(\mathbf{x}) + \boldsymbol{\mu}^T\mathbf{G}(\mathbf{x}),\\
%	&=& \begin{bmatrix}2 & 0\\0 & 0\end{bmatrix} + \boldsymbol{\mu}^T \begin{bmatrix}-2 & 0\\0 & 10\end{bmatrix},
%\end{eqnarray*}

%is a positive definite matrix.

\item Solve the optimality conditions algebraically for the optimal values.

The KKT conditions give the following system of equations,

\begin{eqnarray}	
	\mu_1 + \mu_2 + \mu_3 + \mu_4 &=& 1, \label{eq:kkt1}\\
	10c_x(\mu_1 + \mu_2 + \mu_3 + \mu_4) - 4\mu_1 - 2\mu_2 - 6\mu_3 &=& 0, \label{eq:kkt2}\\
	\mu_1 \left[5c_x^2 - 4c_x + 1 -r^2\right] &=& 0, \label{eq:kkt3}\\
	\mu_2 \left[5c_x^2 - 2c_x + 1 -r^2\right] &=& 0, \label{eq:kkt4}\\
	\mu_3 \left[5c_x^2 - 6c_x + 2 -r^2\right] &=& 0, \label{eq:kkt5}\\
	\mu_4 \left[5c_x^2 -r^2\right] &=& 0, \label{eq:kkt6}\\
	\mu_1 &\ge& 0, \\
	\mu_2 &\ge& 0, \\
	\mu_3 &\ge& 0, \\
	\mu_4 &\ge& 0, \\
	5c_x^2 - 4c_x + 1 -r^2 &\le& 0, \label{eq:ineq1}\\
	5c_x^2 - 2c_x + 1 -r^2 &\le& 0, \label{eq:ineq2}\\
	5c_x^2 - 6c_x + 2 -r^2 &\le& 0, \label{eq:ineq3}\\
	5c_x^2 -r^2 &\le& 0. \label{eq:ineq4}
\end{eqnarray}

In order to solve for the optimal values, we must make assumptions about which constraints are active and which are inactive.
It is clear by inspection of Equation \ref{eq:kkt1} that at least one constraint must be active, and so there is no need to check the possibility of no active constraints.
This is consistent with intuition because if there were no active constraints the solution would trivially be a circle with radius zero centered anywhere on the line $y=2x$.

If we assume all of the constraints are active then the inequalities given by Equations \ref{eq:ineq1}--\ref{eq:ineq4} become equality constraints.
Subtracting Equation \ref{eq:ineq1} from Equation \ref{eq:ineq2} shows that to satisfy both equations, $c_x$ must equal 0.
Plugging $c_x = 0$ into Equation \ref{eq:ineq1} and solving for the radius of the circle yields $r = 1$.
However, plugging $c_x = 0$ into Equation \ref{eq:ineq3} results in, $2 - r^2 \le 0$ which is not true.
This implies the assumption that all constraints are active is not true.
Moreover, this shows that the constraints imposed by the first and second points cannot both be active for solution to be optimal.

Next, let us assume that the constraints imposed by the first point, $(0,1)$, and third point, $(1,1)$, are the only active constraints.
Subtracting Equation \ref{eq:ineq1} from \ref{eq:ineq3} results in $-2c_x + 1 = 0$, or $c_x = 1/2$.
Plugging $c_x = 1/2$ into Equation \ref{eq:ineq1} and solving for the radius of the circle yields $r = 1/2$.
However, $c_x = r = 1/2$ does not satisfy the constraint given by Equation \ref{eq:ineq2}.
This implies the assumption that the constraints imposed by the first and third points are active constraints is not true.

Next, let us assume that the constraints imposed by the second point, $(1,0)$, and third point, $(1,1)$, are the only active constraints.
Subtracting Equation \ref{eq:ineq2} from \ref{eq:ineq3} results in $-4c_x + 1 = 0$, or $c_x = 1/4$.
Plugging $c_x = 1/4$ into Equation \ref{eq:ineq2} and solving for the radius of the circle yields $r = \sqrt{13}/4$.
The values $c_x = 1/2$ and $r = \sqrt{13}/4$ satisfy Equations \ref{eq:ineq1}--\ref{eq:ineq4} as desired.
Since $\mu_1 = \mu_4 = 0$, Equations \ref{eq:kkt1} and \ref{eq:kkt2} become: 

\begin{eqnarray*}
	\mu_2 + \mu_3 &=& 1,\\
	5/2 - 2\mu_2 - 6\mu_3 &=& 0.
\end{eqnarray*}

Solving the system of equations yields $\mu_2 = 7/8$ and $\mu_3 = 1/8$.

The values $\mu_1 = \mu_4 = 0$, $\mu_2 = 7/8$, $\mu_3 = 1/8$, $c_x = 1/4$, and $r = \sqrt{13}/4$ satisfy the KKT conditions.

%Intuition might lead us to consider the constraints imposed by the third and fourth points because the distance between these two points is $\sqrt{2}$ and is greater than or equal to the distance between any other pair of points.
%Subtracting Equation \ref{eq:ineq4} from \ref{eq:ineq3} results in $-6c_x + 2 = 0$, or $c_x = 1/3$.
%Plugging $c_x = 1/3$ into Equation \ref{eq:ineq3} and solving for the radius of the circle yields $r = \sqrt{5}/3$.
%The values $c_x = 1/3$ and $r = \sqrt{5}/3$ satisfy Equation \ref{eq:ineq1}, but not Equation \ref{eq:ineq2}.
%This implies the assumption that the constraints imposed by the first and third points are active constraints is not true.

%Lastly, the optimal solution can be found by considering the constraints imposed by the second and fourth points.
%Subtracting Equation \ref{eq:ineq4} from \ref{eq:ineq2} results in $-2c_x + 1 = 0$, or $c_x = 1/2$.
%Plugging $c_x = 1/2$ into Equation \ref{eq:ineq4} and solving for the radius of the circle yields $r = \sqrt{5}/2$.
%The values $c_x = 1/2$ and $r = \sqrt{5}/2$ satisfy Equations \ref{eq:ineq1}--\ref{eq:ineq4} as desired.
%Since $\mu_1 = \mu_3 = 0$, Equation \ref{eq:kkt2} becomes $5 - 2\mu_2 = 0$ or $\mu_2 = 5/2$.

\item Interpret the meaning of the Lagrange multipliers.

The magnitude of the Lagrange multipliers represents the relative amount of constraint that each active constraint enforces on the optimal solution, i.e. because $\mu_2 = 7/8 > 1/8 = \mu_3$, if the second point is removed (or no longer requires being enclosed) the optimal solution is less than (or more optimal than) if the third point were to be removed.
This was verified.
If the second point is removed, the constraints imposed by the third and fourth points are active.
Algebra yields $c_x = 1/3$, $r = \sqrt{5}/3$, $\mu_3 = 5/9$, and $\mu_4 = 4/9$.
In the case of the third point being removed, the constraint imposed by the second point is the only active constraint.
Algebra yields $c_x = 1/5$, $r = 2\sqrt{5}/5$, $\mu_2 = 1$.
As the circle is larger in the case of the third point being removed, $2\sqrt{5}/5 > \sqrt{5}/3$, this shows that the second point puts more constraint on the optimal solution than the third point does.

\item Is the solution unique?

Yes.

\end{enumerate}

\subsection{Approach 2}

\begin{enumerate}[a)]
	\item Express the problem as an unconstrained optimization problem using the log barrier method.
	
	\begin{equation*}
	\min_{r,c_x} r^2 - \mu \sum_{i=1}^n \log\left[-(x_i - c_x)^2 - (y_i - 2c_x)^2 
	+ r^2\right]
	\end{equation*}
	
	\item Find the optimality conditions.
	
	The first order necessary condition is given by
	
	\begin{equation*}
	\begin{bmatrix}2r \\ 0\end{bmatrix} - \mu \sum_{i=1}^n \left[\frac{1}{-(x_i - c_x)^2 - (y_i - 2c_x)^2 + r^2} \begin{bmatrix}2r \\ 2x_i + 4y_i - 10c_x\end{bmatrix}\right] = \mathbf{0},
	\end{equation*}
	
	or,
	
	\begin{eqnarray*}
	1 - \left[\frac{1}{-5c_x^2 + 4c_x - 1 + r^2} + \frac{1}{-5c_x^2 + 2c_x - 1 +r^2} + \frac{1}{-5c_x^2 + 6c_x - 2 +r^2} + \frac{1}{-5c_x^2 +r^2}\right] &=& 0,\\
	\frac{4-10c_x}{-5c_x^2 + 4c_x - 1 + r^2} + \frac{2-10c_x}{-5c_x^2 + 2c_x - 1 +r^2} + \frac{6-10c_x}{-5c_x^2 + 6c_x - 2 +r^2} + \frac{-10c_x}{-5c_x^2 +r^2} &=& 0.
	\end{eqnarray*}
	
	\item Describe a strategy to produce a feasible initial estimate for this method.
	
	One easy way to produce a feasible initial estimate would be to pick $c_x$ arbitrarily (say $c_x = c_y = 0$) then let $r = \max_i\left[(x_i-c_x)^2 + (y_i-2c_x)^2\right]$.
	This will give a circle with an arbitrary center that encloses all of the points.
	
\end{enumerate}


\newpage

%--- scratch --------- %

Two functions were written in order to solve the smallest triangle problem.
The first was the main function, \texttt{min\_triangle}, which takes a path to a comma separated value (CSV) file as its input and returns $\mathbf{u}$, $d_1$, $d_2$, $d_3$, and the area of the triangle as its output.
The format of the CSV is such that each line contains the coordinates of one of the \textit{n} points, and the coordinates are separated by a comma:

\begin{align*}
x_{11}, & \hspace{0.1in} x_{12}\\
x_{21}, & \hspace{0.1in} x_{22}\\
\vdots & \hspace{0.1in} \vdots\\
x_{n1}, & \hspace{0.1in} x_{n2}
\end{align*}

\noindent If the \texttt{do\_plot} flag that is passed to \texttt{min\_triangle} is true, the function also plots the points given by the CSV and the smallest triangle that encloses them.
The plot helps verify that the solution satisfies the constraints.
This function was run with the sequence of points $\{(3,3), (3,4), (4,4), (5,5), (2,5)\}$ stored in the CSV.

The second function, \texttt{plot\_triangle}, takes $d_1$, $d_2$, and $d_3$ as input and plots an equilateral triangle (using equations (\ref{eq:d1})--(\ref{eq:d3})).
It acts as a helper function for \texttt{min\_triangle}.

\subsubsection{Source Code}
\vspace{0.25in}
\begin{lstlisting}
%% min_triangle.m

function [x, d1, d2, d3, area] = min_triangle(filename, R1, C1, do_plot)
%MIN_TRIANGLE Finds min equilateral triangle that encloses given points.
%
%   [x, d1, d2, d3, area] = MIN_TRIANGLE(filename) finds the smallest
%   equilateral triangle that encloses the set of points contained in 
%   the csv file given by filename. Plots points and resulting triangle.
%
%   [x, d1, d2, d3, area] = MIN_TRIANGLE(filename, R1, C1) finds the 
%	smallest equilateral triangle that encloses the set of points 
%	contained in the csv file given by filename. Skips first R1 rows and
%	C1 columns of csv. Plots points and resulting triangle.
%
%   [x, d1, d2, d3, area] = MIN_TRIANGLE(filename, R1, C1, do_plot) 
%	finds the smallest equilateral triangle that encloses the set of 
%	points contained in the csv file given by filename. Skips first R1 
%	rows and C1 columns of csv. If do_plot is true, plots points and 
%	resulting triangle, else no plot is created.

  validateattributes(filename, {'char','cell'}, {'nonempty'});
  
  if nargin < 3
    R1 = 0;
    C1 = 0;
  end
  
  if nargin < 4
    do_plot = true;
  end
  
  % equations of lines
  a = [0 1; sqrt(3)/2 1/2; -sqrt(3)/2 1/2];
  
  % read points in from csv file
  X = csvread(filename, R1, C1);
  
  % constraints: A * x \leq b
  % d1, d2, d3 = u1 - v1, u2 - v2, u3 - v3
  num_points = size(X, 1);
  A = zeros(3*num_points, 3*num_points+6);
  b = zeros(3*num_points, 1);
  LB = zeros(1, 3*num_points+6);
  UB(1, 1:3*num_points+6) = Inf;
  
  for i=1:num_points
    A(3*i-2,    1:2)   = [1 -1];
    A(3*i-1,    3:4)   = [1 -1];
    A(3*i,      5:6)   = [1 -1];
    A(3*i-2,    3*i+4) =  1;
    A(3*i-1,    3*i+5) = -1;
    A(3*i,      3*i+6) = -1;
    b(3*i-2:3*i)       = a * transpose(X(i,:));
  end
  
  % x = [d1; d2; d3]  = [u1; v1; u2; v2; u3; v3];
  % f = [-1; 1; 1]    = [-1; 1; 1; -1; 1; -1];
  f = zeros(3*num_points + 6, 1);
  f(1:6) = [-1; 1; 1; -1; 1; -1];
  x = linprog(f, [], [], A, b, LB, UB);
  
  d1 = x(1) - x(2);
  d2 = x(3) - x(4);
  d3 = x(5) - x(6);
  
  area = sqrt(3) / 3 * (-d1 + d2 + d3)^2;
  
  if do_plot
    scatter(X(:,1), X(:,2))
    hold on
    plot_triangle(d1, d2, d3);
  end

end
\end{lstlisting}

\subsubsection{Results}

The output of the \texttt{min\_triangle} function was

\vspace{0.25in}
\hrule

\begin{verbatim}
>> [x, d1, d2, d3, area] = min_triangle('points.csv')
Optimization terminated.

x =

  234.1260
  231.1260
  225.3782
  218.5481
  229.3384
  228.5705
  0.0000
  2.7321
  1.8660
  1.0000
  2.2321
  1.3660
  1.0000
  1.3660
  2.2321
  2.0000
  0.0000
  2.5981
  2.0000
  2.5981
  0.0000


d1 =

  3.0000


d2 =

  6.8301


d3 =

  0.7679


area =

  12.2065
\end{verbatim}

\end{document}