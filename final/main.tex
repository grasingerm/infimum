\documentclass{article}

\usepackage{mathtools}
\usepackage{bm}
\usepackage[margin=1.25in]{geometry}
\usepackage{courier}
\usepackage{color}
\usepackage{listings}
\usepackage{pdfpages}
\usepackage{enumerate}

\definecolor{dkgreen}{rgb}{0,0.6,0}
\definecolor{gray}{rgb}{0.5,0.5,0.5}

\newcommand{\blankpage}{
	\newpage
	\thispagestyle{empty}
	\mbox{}
	\newpage
}

\newcommand{\exspace}{\hspace{0.1in}}

\title{Optimization Methods, Final Exam}
\date{2015/12/16}
\author{Matthew Grasinger}

\begin{document}
	
\lstset{language=Matlab,
	keywords={break,case,catch,continue,else,elseif,end,for,function,
		global,if,otherwise,persistent,return,switch,try,while},
	basicstyle=\ttfamily,
	keywordstyle=\color{blue},
	commentstyle=\color{gray},
	stringstyle=\color{dkgreen},
	numbers=left,
	numberstyle=\tiny\color{red},
	stepnumber=1,
	numbersep=10pt,
	backgroundcolor=\color{white},
	tabsize=4,
	showspaces=false,
	showstringspaces=false}
	
\pagenumbering{gobble}
\maketitle
\blankpage
\tableofcontents
\blankpage
\pagenumbering{arabic}

\section{Problem 1} \label{sec:smallest_circle}

\subsection{Objective}

Find the center and radius of the smallest circle that encompasses the following points: $(0,1)$, $(1,0)$, $(1,1)$, $(0,0)$, and whose center lies along the line $y=2x$.

\subsection{Approach 1}

\begin{enumerate}[a)]

\item Express the problem as a constrained minimization problem.

The general equation of a circle can be given as,
\begin{equation} \label{eq:circle_general}
(x - c_x)^2 + (y - c_y)^2 = r^2,
\end{equation}
where $c_x$ and $c_y$ are the $x$ and $y$ coordinates of the center of the circle respectively, and $r$ is the radius of the circle.
The area of a circle is proportional to its radius squared.
Minimizing $r^2$ will be sufficient for finding the smallest circle that encompasses the given points.
By substituting the constraint on the center of the circle into Equation \ref{eq:circle_general}, we obtain the desired constrained minimization formulation,

\begin{align*}
\min_{r, c_x} \exspace & r^2 \\
\text{such that} \exspace & (x_i - c_x)^2 + (y_i - 2c_x)^2 - r^2 \le 0, \exspace \forall i = 1, 2, ..., n
\end{align*}

\item Find the optimality conditions.

The optimality conditions for constrained optimization are given by the so-called KKT conditions.
The KKT conditions for the smallest circle problem are,

\begin{eqnarray*}
\mathbf{\nabla} f(\mathbf{x}) + \sum_{i}^{n} \mu_i \mathbf{\nabla} g_i(\mathbf{x}) &=& \mathbf{0}, \\
\begin{bmatrix}2r \\ 0\end{bmatrix} + \sum_{i}^{n} \mu_i \begin{bmatrix}-2r \\ -2(x_i - c_x) - 4(y_i - 2c_x)\end{bmatrix} &=& \mathbf{0}
\end{eqnarray*}

\item Solve the optimality conditions algebraically for the optimal values.



\item Interpret the meaning of the Lagrange multipliers.



\item Is the solution unique?

\end{enumerate}


%--- scratch --------- %

Two functions were written in order to solve the smallest triangle problem.
The first was the main function, \texttt{min\_triangle}, which takes a path to a comma separated value (CSV) file as its input and returns $\mathbf{u}$, $d_1$, $d_2$, $d_3$, and the area of the triangle as its output.
The format of the CSV is such that each line contains the coordinates of one of the \textit{n} points, and the coordinates are separated by a comma:

\begin{align*}
x_{11}, & \hspace{0.1in} x_{12}\\
x_{21}, & \hspace{0.1in} x_{22}\\
\vdots & \hspace{0.1in} \vdots\\
x_{n1}, & \hspace{0.1in} x_{n2}
\end{align*}

\noindent If the \texttt{do\_plot} flag that is passed to \texttt{min\_triangle} is true, the function also plots the points given by the CSV and the smallest triangle that encloses them.
The plot helps verify that the solution satisfies the constraints.
This function was run with the sequence of points $\{(3,3), (3,4), (4,4), (5,5), (2,5)\}$ stored in the CSV.

The second function, \texttt{plot\_triangle}, takes $d_1$, $d_2$, and $d_3$ as input and plots an equilateral triangle (using equations (\ref{eq:d1})--(\ref{eq:d3})).
It acts as a helper function for \texttt{min\_triangle}.

\subsubsection{Source Code}
\vspace{0.25in}
\begin{lstlisting}
%% min_triangle.m

function [x, d1, d2, d3, area] = min_triangle(filename, R1, C1, do_plot)
%MIN_TRIANGLE Finds min equilateral triangle that encloses given points.
%
%   [x, d1, d2, d3, area] = MIN_TRIANGLE(filename) finds the smallest
%   equilateral triangle that encloses the set of points contained in 
%   the csv file given by filename. Plots points and resulting triangle.
%
%   [x, d1, d2, d3, area] = MIN_TRIANGLE(filename, R1, C1) finds the 
%	smallest equilateral triangle that encloses the set of points 
%	contained in the csv file given by filename. Skips first R1 rows and
%	C1 columns of csv. Plots points and resulting triangle.
%
%   [x, d1, d2, d3, area] = MIN_TRIANGLE(filename, R1, C1, do_plot) 
%	finds the smallest equilateral triangle that encloses the set of 
%	points contained in the csv file given by filename. Skips first R1 
%	rows and C1 columns of csv. If do_plot is true, plots points and 
%	resulting triangle, else no plot is created.

  validateattributes(filename, {'char','cell'}, {'nonempty'});
  
  if nargin < 3
    R1 = 0;
    C1 = 0;
  end
  
  if nargin < 4
    do_plot = true;
  end
  
  % equations of lines
  a = [0 1; sqrt(3)/2 1/2; -sqrt(3)/2 1/2];
  
  % read points in from csv file
  X = csvread(filename, R1, C1);
  
  % constraints: A * x \leq b
  % d1, d2, d3 = u1 - v1, u2 - v2, u3 - v3
  num_points = size(X, 1);
  A = zeros(3*num_points, 3*num_points+6);
  b = zeros(3*num_points, 1);
  LB = zeros(1, 3*num_points+6);
  UB(1, 1:3*num_points+6) = Inf;
  
  for i=1:num_points
    A(3*i-2,    1:2)   = [1 -1];
    A(3*i-1,    3:4)   = [1 -1];
    A(3*i,      5:6)   = [1 -1];
    A(3*i-2,    3*i+4) =  1;
    A(3*i-1,    3*i+5) = -1;
    A(3*i,      3*i+6) = -1;
    b(3*i-2:3*i)       = a * transpose(X(i,:));
  end
  
  % x = [d1; d2; d3]  = [u1; v1; u2; v2; u3; v3];
  % f = [-1; 1; 1]    = [-1; 1; 1; -1; 1; -1];
  f = zeros(3*num_points + 6, 1);
  f(1:6) = [-1; 1; 1; -1; 1; -1];
  x = linprog(f, [], [], A, b, LB, UB);
  
  d1 = x(1) - x(2);
  d2 = x(3) - x(4);
  d3 = x(5) - x(6);
  
  area = sqrt(3) / 3 * (-d1 + d2 + d3)^2;
  
  if do_plot
    scatter(X(:,1), X(:,2))
    hold on
    plot_triangle(d1, d2, d3);
  end

end
\end{lstlisting}

\subsubsection{Results}

The output of the \texttt{min\_triangle} function was

\vspace{0.25in}
\hrule

\begin{verbatim}
>> [x, d1, d2, d3, area] = min_triangle('points.csv')
Optimization terminated.

x =

  234.1260
  231.1260
  225.3782
  218.5481
  229.3384
  228.5705
  0.0000
  2.7321
  1.8660
  1.0000
  2.2321
  1.3660
  1.0000
  1.3660
  2.2321
  2.0000
  0.0000
  2.5981
  2.0000
  2.5981
  0.0000


d1 =

  3.0000


d2 =

  6.8301


d3 =

  0.7679


area =

  12.2065
\end{verbatim}

\end{document}