\documentclass{article}

\usepackage{mathtools}
\usepackage{bm}
\usepackage[margin=1.0in]{geometry}
\usepackage{courier}
\usepackage{color}
\usepackage{listings}

\definecolor{dkgreen}{rgb}{0,0.6,0}
\definecolor{gray}{rgb}{0.5,0.5,0.5}

\title{Optimization Methods, Midterm Exam}
\date{2015/10/28}
\author{Matthew Grasinger}

\begin{document}
	
\lstset{language=Matlab,
	keywords={break,case,catch,continue,else,elseif,end,for,function,
		global,if,otherwise,persistent,return,switch,try,while},
	basicstyle=\ttfamily,
	keywordstyle=\color{blue},
	commentstyle=\color{gray},
	stringstyle=\color{dkgreen},
	numbers=left,
	numberstyle=\tiny\color{red},
	stepnumber=1,
	numbersep=10pt,
	backgroundcolor=\color{white},
	tabsize=4,
	showspaces=false,
	showstringspaces=false}
	
\pagenumbering{gobble}
\maketitle
\newpage
\pagenumbering{arabic}
\tableofcontents
\newpage

\section{Problem 1: The Smallest Triangle}

\subsection{Objective}

The objective of the problem is to find the smallest equilateral triangle that encloses a given set of \textit{n} points in two dimensions, $(x_1, x_2)$.
The base of the triangle is fixed to be parallel to the $x_1$-axis.
The base of the triangle is below the points.

\subsection{Formulation}

We can find a solution to the smallest triangle problem by formulating it as a linear program.
This consists of describing the problem by constraints as a linear system of equations and a cost function that is linear in the unknowns.

\subsubsection{Constraints}

A triangle consists of three lines where each pair of lines meets at a vertex.
For the lines of an equilateral triangle where the base is parallel to the $x_1$-axis, the lines can be given by the three equations,

\begin{equation*}x_2 = d_1\end{equation*}
\begin{equation} \label{eq:triangle_lines_ls}
\frac{\sqrt{3}}{2} x_1 + \frac{1}{2} x_2 = d_2
\end{equation}
\begin{equation*}-\frac{\sqrt{3}}{2} x_1 + \frac{1}{2} x_2 = d_3\end{equation*}

\noindent where $d_1$, $d_2$, and $d_3$ are the three unknowns that construct the triangle.
Alternatively, Equation \ref{eq:triangle_lines_ls} can be written as:

\begin{equation} \label{eq:triangle_lines_la}
\mathbf{a}_i^T \mathbf{x} = d_i
\end{equation}

\noindent where $\mathbf{a}_1 = \begin{bmatrix}0 && 1\end{bmatrix}^T$, $\mathbf{a}_2 = \begin{bmatrix}\frac{\sqrt{3}}{2} && \frac{1}{2}\end{bmatrix}^T$, $\mathbf{a}_3 = \begin{bmatrix}-\frac{\sqrt{3}}{2} && \frac{1}{2}\end{bmatrix}^T$ and $\mathbf{x} = \begin{bmatrix}x_1 && x_2\end{bmatrix}^T$.

Now that we have equations that construct a unique triangle, it is necessary to introduce constraints on $d_1$, $d_2$, and $d_3$.
In plain English, we need ensure that the triangle is constructed such that all of the points lie inside of it.
To express this formally, we need to establish relationships between each point and each line.

\section{MATLAB Code}

The following matlab script was written and ran in order to solve the river crossing problem:

\begin{lstlisting}
\end{lstlisting}

\section{Results}

The output of the script was:

\begin{verbatim}
2 = the number of forward trips alpha makes with beta
1 = the number of forward trips gamma makes with delta
1 = the number of backward trips alpha makes
1 = the number of backward trips beta makes
The total time taken for everyone to cross = 17
\end{verbatim}

\end{document}